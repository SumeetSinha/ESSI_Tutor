\documentclass{article}
\usepackage[a4paper, left=30mm,  right=30mm]{geometry}
\usepackage{devanagari}

\title{Real ESSI Simulator Tutorial}
\author{Sumeet Kumar Sinha \\ {\dn \7{s}Emt \7{k}mAr Es\306whA }  }
\date{16th April, 2015}

\begin{document}

\maketitle
ESSI is an acronym for Earthquake-Soil-Structure-Interaction Simmulator. The language was developed in C by Jose Abel, a PhD student of Prof. Boris Jeremic at University of California Davis. The primary aim was to develop FEA models and make them interface with various ESSI functionalities. It has been mainely designed and developed for parallel programming to perform high speed calculations in simulation of large models while a sequential version has also been made available. The Real ESSI Simulator systems consist of the Real ESSI Program, Real ESSI Computer and Real ESSI Notes. Pronounciation of Real ESSI is simalar to "real easy" of which it's translation becomes {\dn b\7{h}t hF aAsAn} in Hindi. 
\\
\\
\textbf{Note:} The literal meaning of Real is "{\dn aslF}" and ESSI is "{\dn aAsAn}" but with specific reference to the above subject it would be "{\dn b\7{h}t hF aAsAn}" meaning "extremely eassy".

\section {Getting Started} 
since the language was developed in C using YACC and LEX tool, a lot of similarities can be found particular to the syntax of ESSI. It's command grammer structure and wording provides a powerfull self-documenting syntax which ensuresd that the resulting model script is readable and understandable with little or non reference to user manual. A quick look at other features of the program are:

\begin{itemize}
  \item[$\bullet$] Finite Element Analysis (FEA) is unitless, i.e. the calculations are carried out without any reference unit system. It is the user who should take care of the correctness of units. This feature ensures the user to keep an eye on units enabling them to catch the common mistakes which usually mess up with the results.  
  \item[$\bullet$] It provides modularity through the include directive/commands as in C which allows complex analysis cases to be parametrised into modules and functions which can be reused agin in other model and make the program more compact and understandable.
  \item[$\bullet$] It provides an interactive programming environment with all ESSI syntax available. Also several standard tools are provided to check element validity (jacobian, nodes, .. etc)
  \item[$\bullet$] Last, it provides reduced model developmentg time by providing teh aforementioned features along with meaningful error reporting (of syntax and grammatical errors), a help system, command completion and highlighting for several open source and commercial text editors.
\end {itemize}

\subsection{Installlation Process}


\section {Basic Syntax}
\section {}
\end{document}
