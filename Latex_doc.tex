\documentclass{article}
\usepackage[a4paper, left=30mm,  right=30mm]{geometry}
\usepackage{listings}
\usepackage{xcolor}
\usepackage{color}
\usepackage{devanagari}

\definecolor{dkgreen}{rgb}{0,0.6,0}
\definecolor{gray}{rgb}{0.5,0.5,0.5}
\definecolor{mauve}{rgb}{0.58,0,0.82}
\definecolor{ltgray}{rgb}{0.98,0.98,0.98}
\definecolor{grayish}{rgb}{0.85, 0.85, 0.85}

\lstset{frame=single,
  backgroundcolor=\color{ltgray},
  aboveskip=3mm,
  belowskip=3mm,
  showstringspaces=false,
  columns=flexible,
  basicstyle={\small\ttfamily},
  numberstyle=\tiny\color{black},
  keywordstyle=\color{blue},
  commentstyle=\color{gray},
  stringstyle=\color{mauve},
  breaklines=true,
  numbers=left,
  breakatwhitespace=true,
  tabsize=3,
  %rulecolor=\color{mauve},
}

\title{Real ESSI Simulator Tutorial}
\author{Sumeet Kumar Sinha \\ {\dn \7{s}Emt \7{k}mAr Es\306whA }  }
\date{\today}

\begin{document}

\maketitle
ESSI is an acronym for Earthquake-Soil-Structure-Interaction Simmulator. The language was developed in C by Jose Abel, a PhD student of Prof. Boris Jeremic at University of California Davis. The primary aim was to develop FEA models and make them interface with various ESSI functionalities. It has been mainely designed and developed for parallel programming to perform high speed calculations in simulation of large models while a sequential version has also been made available. The Real ESSI Simulator systems consist of the Real ESSI Program, Real ESSI Computer and Real ESSI Notes. Pronounciation of Real ESSI is simalar to "real easy" of which it's translation becomes {\dn b\7{h}t hF aAsAn} in Hindi. 
\\
\\
\textbf{Note:} The literal meaning of Real is "{\dn aslF}" and ESSI is "{\dn aAsAn}" but with specific reference to the above subject it would be "{\dn b\7{h}t hF aAsAn}" meaning "extremely eassy".

\section {Getting Started} 
since the language was developed in C using YACC and LEX tool, a lot of similarities can be found particular to the syntax of ESSI. It's command grammer structure and wording provides a powerfull self-documenting syntax which ensuresd that the resulting model script is readable and understandable with little or non reference to user manual. A quick look at other features of the program are:

\begin{itemize}
  \item[$\bullet$] Finite Element Analysis (FEA) is unitless, i.e. the calculations are carried out without any reference unit system. It is the user who should take care of the correctness of units. This feature ensures the user to keep an eye on units enabling them to catch the common mistakes which usually mess up with the results.  
  \item[$\bullet$] It provides modularity through the include directive/commands as in C which allows complex analysis cases to be parametrised into modules and functions which can be reused again in other model and make the program more compact and understandable.
  \item[$\bullet$] It provides an interactive programming environment with all ESSI syntax available. Also several standard tools are provided to check element validity (jacobian, nodes, .. etc). Users can query nodes and elements to see if they are set to appropritae states.
  \item[$\bullet$] Last, it provides reduced model developmentg time by providing tech aforementioned features along with meaningful error reporting (of syntax and grammatical errors), a help system, command completion and highlighting for several open source and commercial text editors.
\end {itemize}

\subsection{Installlation Process}
Installation commands for Linux distribution systems (for Ubuntu) are given below.The whole installation steps for both Sequential and Paraller versions can be divided into three subsections. 

\subsubsection{Updating GCC compiler version to 4.9}
ESSI uses latest gcc compiler version 4.9. It is strongly suggested to update the compiles and listers.

\lstinputlisting[language=bash, frame=single]{Compiler.sh}

\subsubsection{Installing Libraries and Packages}
ESSI requires a development version of BOOST C++ library ( http://www.boost.org ) and special external packages made for essi to be installedon the system.\\

\textbf{Installing BOOST C++ Library}  

\lstinputlisting[language=bash, frame=single]{Boost.sh}

\textbf{Installing Packages/libraries for ESSI}  

\lstinputlisting[language=bash, frame=single]{EssiLib.sh}

\subsubsection{Setup for Sequential/Parallel Version}
This section provides the final setup for started ESSI in sequential as well as parallel environment.\\

\textbf{Sequential Version - }
Sequential compilation of code is nothing new and has been there for since 1990's. To configure the sequential version, follow the steps as shown below.

\lstinputlisting[language=bash, frame=single]{Boost.sh}

\textbf{Parallel Version - }
In addition to above general packages, for parallel version separate packages must be installed on ubuntu which are descibed in the code below. Compilation in parallel is done using available CPUs/cores (which can be defined in \textbf{make.sh} file). It is important to note that parallel compilation assumes a shared memory parallel (SMP) computer, where all the CPUs have access to the same memory pool. Number of CPUs will only change/improve the speed of compilation, while for building a parallel version of the Real ESSI Program, different approach is used (to be described later). Main executable program for the Real ESSI Simulator is in folder bin and is named essi. Shell script clean.sh removes all the object files for all libraries, while leaving the main executable essi untouched.

\lstinputlisting[language=bash, frame=single]{Boost.sh}

\subsection {Running ESSI}
ESSI's interactive mode can be invoked from the bash terminal by typing \textbf{essi}. 

\lstinputlisting[language=bash, frame=single,backgroundcolor=\color{grayish}]{ESSI_interactive_mode}

\noindent From here commands can be input manually or a file may be included via the \textbf{include} command.

\begin{lstlisting}[language=C,backgroundcolor=\color{grayish}]
ESSI> include "foobar.fei";                         // includes the file foobar.fei
\end{lstlisting}

\noindent A file can be executed immediately through the command line \textbf{ESSI} by issuing \textbf{essi -f foobar.fei} to execute essi with foobar.fei as input. After executing the file, the interpreter will continue in interactive mode unless the command line \textbf{flag -n} or \textbf{ \textbf{-}-no-interactive} is set. A list of command line options can be seen by calling essi from the command line as \textbf{essi -h}.

\lstinputlisting[language=bash, frame=single,backgroundcolor=\color{grayish}]{ESSI_interactive_commands}

\noindent To exit \textbf{ESSI}, just type \textbf{bye;} in the prompt. 

\begin{lstlisting}[language=C,backgroundcolor=\color{grayish}]
ESSI> bye;                        
\end{lstlisting}

\noindent  If \textbf{bye;} command is included at the end of a script, the essi program will exit upon execution of the script regardless of any errors which occur during execution.


\section {ESSI Basics - Domain Specific Language}
ESSI has its own semantics ( \textbf{DSL - Domain Specific Language} ) developed using Yacc and Lex to make it more simplistic while preserving the traditional coding paradigms.
   
\subsection{ESSI Baisc Syntax}

\begin{itemize}
\item[$\bullet$]Each command line end with a \textbf{semicolon} (\textbf{;})
\item[$\bullet$]Line Comments begin with either \textbf{//} or \textbf{!}
\item[$\bullet$]\textbf{Include} statements allow splitting into several files
\item[$\bullet$]All variables are \textbf{double precision floats} and has a unit attached to it
\item[$\bullet$]All standard \textbf{arithmetic operations} are implemented and are unit sensitive
\item[$\bullet$]Internally, all variables are represented in standard units \textbf{[M L T]} or \textbf{[m s kg]} 
\item[$\bullet$]Extra white spaces, tabulation does not affect the code but can be used to make the code readable and appear aesthetic. 
\item[$\bullet$]Every model must have a name : \textbf{model name} 
\textit{$<$"model\_name\_string"$>$}
\end{itemize}

\subsection{ESSI Datatypes}
\textbf{ESSI} has basically three datatypes: \textbf{float}, \textbf{integer} and \textbf{string}. \textbf{Double precision float} is the most common datatype and quantifies every variable declard in ESSI. While \textbf{integer} and \textbf{string} are only used as a parameter for predefined functions/objects inside ESSI.

\begin{lstlisting}[language=C]
length = 7.6e-31;
new loading stage "loading_stage_name-string";          // # string as loading stage name
remove element # 1;                                     // # <element no> is a positive integer                      
\end{lstlisting}

\begin{itemize}
\item[$\bullet$]\textbf{Float} datatype is by default double-precision 64-bit IEEE 754 floating point. 
\item[$\bullet$]\textbf{Integers} are used to number the ESSI objects like nodes, elements, loads, contraints etc.
\item[$\bullet$]\textbf{Strings} are similarly used as Integers to name stages, models, materials etc.
\item[$\bullet$]ESSI also has functions that return boolean but of the form \textbf{1 and 0} meaning true and false.  
\end{itemize}

\begin{lstlisting}[language=C]
if (isForce(L1))                  // isForce(L1) - will either return 0 or 1
{                                 // 1 -> if L1 is a force quantity
 print L1 ;                       // 0 -> if L1 is not a force quantity
}                                 // print L1 - prints the value of variable L1
\end{lstlisting}

\subsection{ESSI Variables}

A \textbf{ESSI variable} is a named storage location which can be accessed anytime/anywhere by/in our programs by just refering to it by its name. All ESSI variables are of double precision float datatype and are defined using assignment (\textbf{=}) operator.

\begin{lstlisting}[language=C]
 x = 7;                           // variable x is set to 7 which is ( unitless )
 Modulus = 200e10;                // scientific notation is available
 Length = 1.05*m                  // Length is set to 1.05 m (metres)
\end{lstlisting}

\noindent An important distinction of  \textbf{ESSI variable} is that they can have \textbf{units} associated with them like kg, m, mm, Pa etc. which can be seen in ESSI Units section.
ESSI also has some \textbf{predefined locked variables} which act as reference or name to standard units.

\begin{lstlisting}[language=Python]
print x;
print Length;
\end{lstlisting}

\noindent The \textbf{print} command displays the current value of a variable.

\begin{lstlisting}[language=C,backgroundcolor=\color{grayish}]
x = 7 []
Length = 1.05 [m]
\end{lstlisting}

\noindent Here, the \textbf{[]} sign means that the quantities rae unitless. The command \textbf{whos} can be used to see all the currently defined variables and their values.

\begin{lstlisting}[language=C,backgroundcolor=\color{grayish}]
ESSI > whos;

Declared variables:
  * GPa = 1 [GPa]
  * Hz = 1 [Hz]
  * Km = 1 [km]
    Length = 1.05 [m]
  * MPa = 1 [MPa]
  * N = 1 [N]
  * Pa = 1 [Pa]
  * cm = 1 [cm]
  * g = 9.81 [m*s^-2]
  * kN = 1 [kN]
  * kPa = 1 [kPa]
  * kg = 1 [kg]
  * m = 1 [m]
  * mm = 1 [mm]
  * ms = 1 [ms]
  * ns = 1 [ns]
  * s = 1 [s]
    x = 7 []

  * = locked variable
\end{lstlisting}


\subsection{ESSI Units}

\subsection{ESSI Operations}

\subsection{ESSI Decision Making}

\subsection{ESSI Input Files}

\subsection{ESSI Command List}

\section {Building Stages}
\section {Building Material Model}
\section {Modelling Nodes}
\section {Modelling Finite Element}
\section {Modelling Constraints}
\section {Modelling Damping}
\section {Modelling Static Loads}
\section {Modelling Dynamic Loads}
\section {Modelling prescribed Displacements}
\section {Modelling Prescribed Displacements}
\section {Modelling Simulation}
\section {Modelling Output}

\end{document}
