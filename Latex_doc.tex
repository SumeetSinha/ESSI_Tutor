\documentclass{article}
\usepackage[a4paper, left=30mm,  right=30mm]{geometry}
\usepackage{listings}
\usepackage{xcolor}
\usepackage{color}
\usepackage{devanagari}

\definecolor{dkgreen}{rgb}{0,0.6,0}
\definecolor{gray}{rgb}{0.5,0.5,0.5}
\definecolor{mauve}{rgb}{0.58,0,0.82}
\definecolor{ltgray}{rgb}{0.98,0.98,0.98}

\lstset{frame=single,
  backgroundcolor=\color{ltgray},
  aboveskip=3mm,
  belowskip=3mm,
  showstringspaces=false,
  columns=flexible,
  basicstyle={\small\ttfamily},
  numberstyle=\tiny\color{black},
  keywordstyle=\color{blue},
  commentstyle=\color{gray},
  stringstyle=\color{mauve},
  breaklines=true,
  numbers=left,
  breakatwhitespace=true,
  tabsize=3,
  %rulecolor=\color{mauve},
}

\title{Real ESSI Simulator Tutorial}
\author{Sumeet Kumar Sinha \\ {\dn \7{s}Emt \7{k}mAr Es\306whA }  }
\date{\today}

\begin{document}

\maketitle
ESSI is an acronym for Earthquake-Soil-Structure-Interaction Simmulator. The language was developed in C by Jose Abel, a PhD student of Prof. Boris Jeremic at University of California Davis. The primary aim was to develop FEA models and make them interface with various ESSI functionalities. It has been mainely designed and developed for parallel programming to perform high speed calculations in simulation of large models while a sequential version has also been made available. The Real ESSI Simulator systems consist of the Real ESSI Program, Real ESSI Computer and Real ESSI Notes. Pronounciation of Real ESSI is simalar to "real easy" of which it's translation becomes {\dn b\7{h}t hF aAsAn} in Hindi. 
\\
\\
\textbf{Note:} The literal meaning of Real is "{\dn aslF}" and ESSI is "{\dn aAsAn}" but with specific reference to the above subject it would be "{\dn b\7{h}t hF aAsAn}" meaning "extremely eassy".

\section {Getting Started} 
since the language was developed in C using YACC and LEX tool, a lot of similarities can be found particular to the syntax of ESSI. It's command grammer structure and wording provides a powerfull self-documenting syntax which ensuresd that the resulting model script is readable and understandable with little or non reference to user manual. A quick look at other features of the program are:

\begin{itemize}
  \item[$\bullet$] Finite Element Analysis (FEA) is unitless, i.e. the calculations are carried out without any reference unit system. It is the user who should take care of the correctness of units. This feature ensures the user to keep an eye on units enabling them to catch the common mistakes which usually mess up with the results.  
  \item[$\bullet$] It provides modularity through the include directive/commands as in C which allows complex analysis cases to be parametrised into modules and functions which can be reused agin in other model and make the program more compact and understandable.
  \item[$\bullet$] It provides an interactive programming environment with all ESSI syntax available. Also several standard tools are provided to check element validity (jacobian, nodes, .. etc)
  \item[$\bullet$] Last, it provides reduced model developmentg time by providing teh aforementioned features along with meaningful error reporting (of syntax and grammatical errors), a help system, command completion and highlighting for several open source and commercial text editors.
\end {itemize}

\subsection{Installlation Process}
Installation commands for Linux distribution systems (for Ubuntu) are given below.The whole installation steps for both Sequential and Paraller versions can be divided into three subsections. 

\subsubsection{Updating GCC compiler version to 4.9}
ESSI uses latest gcc compiler version 4.9. It is strongly suggested to update the compiles and listers.

\lstinputlisting[language=bash, frame=single]{Compiler.sh}

\subsubsection{Installing Libraries and Packages}
ESSI requires a development version of BOOST C++ library ( http://www.boost.org ) and special external packages made for essi to be installedon the system.\\

\textbf{Installing BOOST C++ Library}  

\lstinputlisting[language=bash, frame=single]{Boost.sh}

\textbf{Installing Packages/libraries for ESSI}  

\lstinputlisting[language=bash, frame=single]{EssiLib.sh}

\subsubsection{Setup for Sequential/Parallel Version}
This section provides the final setup for started ESSI in sequential as well as parallel environment.\\

\textbf{Sequential Version - }
Sequential compilation of code is nothing new and has been there for since 1990's. To configure the sequential version, follow the steps as shown below.

\lstinputlisting[language=bash, frame=single]{Boost.sh}

\textbf{Parallel Version - }
In addition to above general packages, for parallel version separate packages must be installed on ubuntu which are descibed in the code below. Compilation in parallel is done using available CPUs/cores (which can be defined in \textbf{make.sh} file). It is important to note that parallel compilation assumes a shared memory parallel (SMP) computer, where all the CPUs have access to the same memory pool. Number of CPUs will only change/improve the speed of compilation, while for building a parallel version of the Real ESSI Program, different approach is used (to be described later). Main executable program for the Real ESSI Simulator is in folder bin and is named essi. Shell script clean.sh removes all the object files for all libraries, while leaving the main executable essi untouched.

\lstinputlisting[language=bash, frame=single]{Boost.sh}


\section {Basic Syntax}

\begin{itemize}
\item[$\bullet$]Each command line end with a \textbf{semicolon} (\textbf{;})
\item[$\bullet$]Comments begin with either // or ! and last until the end of the line
\end{itemize}

\section {Building Stages}
\section {Building Material Model}
\section {Modelling Nodes}
\section {Modelling Finite Element}
\section {Modelling Constraints}
\section {Modelling Damping}
\section {Modelling Static Loads}
\section {Modelling Dynamic Loads}
\section {Modelling prescribed Displacements}
\section {Modelling Prescribed Displacements}
\section {Modelling Simulation}
\section {Modelling Output}

\end{document}
